\begin{center}
  \textsc{Abstract}
\end{center}
%
\noindent
%
As agricultural inputs, such as water and fertilizer, become scarce a higher level of precision will be required when utilizing these resources in food production. Remote sensing is a field that aims to understand data from radiation reflected from vegetation.  Determination of water content with remote sensing techniques
of vegetation remains a long term goal.  Furthermore as agricultural production has shifted to indoor
growing environments, the use of indoor sensors at smaller scales may prove to be useful for determining the
physiological status of crops. In this study the use of texture, polarization and pseudo-spectral features captured in
the acquired images are shown to be useful for the successful classification of
three different deciduous tree species common to the northeastern part of the
United States using a linear support vector classifier. The observations
are extended to the intra-class variance of the derived features which are shown
to be useful for the prediction of the relative water content of individual leaves when
analyzed using linear regression in the specular direction.
