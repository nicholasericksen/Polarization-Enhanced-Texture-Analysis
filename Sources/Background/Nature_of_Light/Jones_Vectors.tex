%%%%%%%%%%%%%%%%%%%%%%%%%%%%%%%%%%%%%%%%%%%%%%%%%%%%%%%%%%%
\subsection{Jones Vector Representation}
%%%%%%%%%%%%%%%%%%%%%%%%%%%%%%%%%%%%%%%%%%%%%%%%%%%%%%%%%%%
    %need subsub section on Jones
    % need subsub section on Devices
For the special case of fully polarized EM waves, DOP = 1, the polarization of the beam can be described by a 2x1 complex vector known as a Jones vector.  The Jones vector relies on the fact that the polarization state of a fully polarized beam depends only on its relative $x$ and $y$ intensities, as well as the phase delay between each respective component.

Converting the equation for the electric component of the EM wave into a phasor makes it easy to see the parameters that determine the beam's polarization.  A phasor represents a sinusoidal wave with a constant frequency.  The Jones vectors can be formulated from this representation as
%
\begin{align}
    \underline{\hat{E}}(z)=(\vec{i_x} E_{0x} e^{j\delta_x}+\vec{i_y} E_{0y} e^{j\delta_y })e^{-kz}
\end{align}
%
since the polarization depends on the amplitude and phase difference of the $x$ and $y$ components, the Jones vector is formally written as
%
\begin{align}
    \underline{J} =
    \begin{bmatrix}
        E_{0x} e^{j\delta_x} \\
        E_{0y} e^{j\delta_y }
    \end{bmatrix}
\end{align}
%
Since only the relative phase differences matter it is common to denote $\delta=\delta_y-\delta_x$.  The vector is also normalized by dividing by its magnitude,
%
\begin{align}
    \underline{J} =
    \frac{1}{\sqrt{E_{0x}^2 + E_{0y}^2}}
    \begin{bmatrix}
        E_{0x} e^{j\delta_x} \\
        E_{0y} e^{j\delta_y }
    \end{bmatrix}
\end{align}
%
An angle can then be defined such that
%
\begin{align}
    tan(\psi) = \frac{E_{0y}}{E_{0x}}
\end{align}
%
The Jones vector can then be written as
%
\begin{align}
    \underline{J_{\delta}}(\psi) =
    \begin{bmatrix}
        cos(\psi) \\
        sin(\psi)e^{j\delta}
    \end{bmatrix}
\end{align}
%
General states of linear polarization are represented when the phase delay between the $x$ and $y$ components are zero. This is shown in general as
%
\begin{align}
    \underline{J_0}(\psi) =
    \begin{bmatrix}
        cos(\psi) \\
        sin(\psi)e^{j*0}
    \end{bmatrix}
    =
    \begin{bmatrix}
        cos(\psi) \\
        sin(\psi)
    \end{bmatrix}
\end{align}
%
where $\psi$ is any angle in relation to the X axis. For horizontal and vertical polarization it can be shown that
%
\begin{align}
    \underline{J_0}(0) = \underline{J_H} =
    \begin{bmatrix}
        cos(0) \\
        sin(0)
    \end{bmatrix}
    =
    \begin{bmatrix}
        1 \\
        0
    \end{bmatrix}
\end{align}
\begin{align}
    \underline{J_0}(90) = \underline{J_V} =
    \begin{bmatrix}
        cos(\frac{\pi}{2}) \\
        sin(\frac{\pi}{2})
    \end{bmatrix}
    =
    \begin{bmatrix}
        0 \\
        1
    \end{bmatrix}
\end{align}
Circular polarization is represented as
%
\begin{align}
    RCP: \underline{J}_{\frac{\pi}{2}} = \frac{1}{\sqrt{2}}
    \begin{bmatrix}
        1 \\
        j
    \end{bmatrix} \\
    LCP: \underline{J}_{-\frac{\pi}{2}} = \frac{1}{\sqrt{2}}
    \begin{bmatrix}
        1 \\
        -j
    \end{bmatrix}
\end{align}
%
\subsubsection{Optical Devices}
Polarization can be naturally occurring, such as in the case of skylight, or it can be created by passing light through an optical device such as a linear polarizer or a quarter wave plate.  Jones vectors are useful for describing the polarization state of an EM wave, while Jones matrices describe nondepolarizing optical devices and the transformation of pure incident polarization states through them.

A linear polarizer is a device that transmits linear polarization states for incident light beams that are aligned with their transmission axis (TA) of the polarizer \cite{polarizedlight}.  For example, if horizontally polarized light is passed through a polarizer with a $TA = 90^{\circ}$, all of the incident light will be extinguished.  In practice all of the light is not completely extinguished and there are often spectral differences to the response of polarizers.

Since linear polarizers block light that is orthogonal to the TA, it can be shown that the general equation for a linear polarizer is such that
%
\begin{align}
    \underline{J}_{in}(\psi + \frac{\pi}{2}) =
    \begin{pmatrix}
        cos(\psi + \frac{\pi}{2}) \\
        sin(\psi + \frac{\pi}{2})
    \end{pmatrix}
    =
    \begin{pmatrix}
        -sin(\psi) \\
        cos(\psi)
    \end{pmatrix}
\end{align}
%
and
\begin{align}
    \underline{J}_{out} =
    \begin{pmatrix}
        0 \\
        0
    \end{pmatrix}
\end{align}
%
The general equation for Jones interaction with an orthogonal linear polarizer is
%
\begin{align}
    P(\psi)\underline{J}_{in} = \underline{J}_{out}
\end{align}
\begin{align}
    \begin{pmatrix}
        a & b \\
        c & d
    \end{pmatrix}
    \begin{pmatrix}
        -sin(\psi) \\
        cos(\psi)
    \end{pmatrix}
    =
    \begin{pmatrix}
        0 \\
        0
    \end{pmatrix}
\end{align}
\begin{align}
    -asin(\psi) + bcos(\psi) = 0 \rightarrow atan(\psi) = b \\
    -csin(\psi) = dcos(\psi) = 0 \rightarrow ctan(\psi) = d
\end{align}
%
When the incident polarization state is aligned with the polarizers TA, it must also be true that the incident beam goes through the device unchanged.  Therefore,
%
\begin{align}
    \underline{J}_{in}(\psi) =
    \begin{pmatrix}
        cos(\psi) \\
        sin(\psi)
    \end{pmatrix} \\
    \underline{J}_{out}(\psi) =
    \begin{pmatrix}
        cos(\psi) \\
        sin(\psi)
    \end{pmatrix}
\end{align}
\begin{align}
    acos(\psi) + bsin(\psi) = cos(\psi) \\
    ccos(\psi) + dsin(\psi) = sin(\psi)
\end{align}
%
substituting in for previous values of b and d give
%
\begin{align}
    a = \frac{cos(\psi)}{cos(\psi) + tan(\psi)sin(\psi)} = cos^2\psi
\end{align}
\begin{align}
    b = atan(\psi) = sin(\psi)cos(\psi)
\end{align}
\begin{align}
    c = \frac{sin(\psi)}{cos(\psi) + tan(\psi)sin(\psi)} = sin(\psi)cos(\psi)
\end{align}
\begin{align}
    d = ctan(\psi) = sin^2\psi
\end{align}
%
The general form of a linear polarizer with transmission axis angle $\psi$ from the X axis is
%
\begin{align}
    P(\psi) =
    \begin{pmatrix}
        cos^2\psi & sin(\psi)cos(\psi) \\
        sin(\psi)cos(\psi) & sin^2\psi
    \end{pmatrix}
\end{align}
%
The intensity of light emerging from a polarizer is governed by Malus law,
%
\begin{align}
    I = I_0cos^2(\theta_i)
\end{align}
%
where $I$ is the intensity of the exiting beam, $I_0$ is the intensity of the incident beam and $\theta_i$ is the angle between the incident polarization state, and the angle of the polarizer.   For incident unpolarized light the equation becomes $I / I_0 = \frac{1}{2}$.  Therefore, the maximum transmittance for an unpolarized beam of light through a polarizer is \text{50\%}.  This number is significantly lower for polaroid film based polarizers and can cause potential problems with detector sensitivity for low lighting conditions.

Wave plates create a phase delay between the fast and slow axis of incident linearly polarized light.  Its generalized Jones matrix form can be denoted with a relative phase delay $\delta=\delta_y-\delta_x$,
\begin{align}
    C(\delta) =
    \begin{pmatrix}
        1 & 0 \\
        0 & e^{-j\delta}
    \end{pmatrix}
\end{align}
%
Two common wave plates are the half wave plate and the quarter wave plate.  These produce a delay of $\pi$ and $\pi/2$ respectively.

The noobelectric Python package has a module for dealing with these types of problems and it automates Jones optical calculations for purely polarized light beams.  An example for a linear polarizer combined with a quarter wave plate can be created and various input polarization states into the system as shown below.
\begin{figure}
    \begin{lstlisting}
        >>> from noobee.jones import j, lp
        >>> j_in = j(1,0)
        >>> lp_1 = lp(0)
        >>> lp_1 * j_in
        matrix([[ 1.],
                [ 0.]])
    \end{lstlisting}
    \caption{noobee code for Jones Vectors}
    \label{fig:scattering}
\end{figure}
