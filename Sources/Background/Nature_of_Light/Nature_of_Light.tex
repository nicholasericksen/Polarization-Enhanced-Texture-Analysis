%%%%%%%%%%%%%%%%%%%%%%%%%%%%%%%%%%%%%%%%%%%
\section{The Nature of Light and the Electromagnetic Spectrum}
%%%%%%%%%%%%%%%%%%%%%%%%%%%%%%%%%%%%%%%%%%%
Light is fundamental to life.  In Maxwell’s’ theory on light he explained how electromagnetic phenomenon can be expressed in terms of waves [6].  These waves move at the speed of light.  The light that human eyes can detect is a small portion of the electromagnetic spectrum called ‘visible light’.

[insert EM image here]
Different frequencies of light correspond to different colors on the visible spectrum.  These waves are made up of two parts, the electric and magnetic.  Due to their similar nature often only the electrical portion is considered at first for mathematical simplicity.

The electrical portion of EM waves, and the only portion considered here, can be described as the sum of two sinusoidal waves representing the orthogonal x and y components in Cartesian space.
%
\begin{align}
    E(z,t)=E_x (t)+E_y (t)\\
    E_x (t)=E_{0x} cos( \theta+\delta_x )\\
    E_y (t)=E_{0y} cos( \theta+\delta_y )
\end{align}
%
where $ \theta = \delta t-kz $ is the wave propagator that determines the frequency and direction of propagation for the wave.$ \delta_y and \delta_x $ represent the phase delay for each component of the wave.

When more than one sine wave is considered, as in the case of EM waves, an overall phase delay between the them is considered and represented as $ \tau=\delta_y-\delta_x $.

Light can also be viewed as packets of energy known as photons.  The energy of photons is related to its frequency $v$ and a constant, known as Planck’s constant $h$ [9].
%
\begin{align}
	h=6.62e-34 [m^2 kg/s]\\
	E=hv
\end{align}
%
This simple equation shows that for higher frequencies, particles have higher energy.

%%%%%%%%%%%%%%%%%%%%%%%%%%%%%%%%%%%%%%%%%%%%%%%%%%%%%%%%%%%%%%
% Nature of Light Subsections
%%%%%%%%%%%%%%%%%%%%%%%%%%%%%%%%%%%%%%%%%%%%%%%%%%%%%%%%%%%%%%
%%%%%%%%%%%%%%%%%%%%%%%%%%%%%%%%%%%%%%%%%%%
\subsection{Polarization of EM Waves}
%%%%%%%%%%%%%%%%%%%%%%%%%%%%%%%%%%%%%%%%%%%
The polarization of EM waves is determined, for monochromatic frequencies, by the relative intensity and phase of their respective $x$ and $y$ components.  These relationships can be viewed as the path traced by the tip of the electric field vector when looking in the direction of illumination.  Common sources of illumination are lasers, light emitting diodes, halogen lamps, the sun, etc.

In its most general form, the polarization is referred to as being elliptical, and its $x$ and $y$ amplitudes, and phase delay can be described in the form of the polarization ellipse.

It has been shown that the form of the polarization ellipse can be derived from the solution to the plane wave equation for the electromagnetic wave.  Using the relationships defined in the previous section and defining,
%
\begin{align}
    \tau=\omega t-kz+\delta_x
\end{align}
%
We can then define the $x$ and $y$ component of the wave as
%
\begin{align}
	E_x (t)=E_{0x} cos(\tau)\\
	E_y (t)=E_{0y} cos(\tau+\delta )
\end{align}
%
Dividing each equation by its intensity results in
%
\begin{align}
    \frac{E_x (t)}{E_{0x}} =cos(\tau) \\
    \frac{E_y (t)}{E_{0y}} =cos(\tau+\delta )
\end{align}
%
The $y$ component is then separated using known trigonometric identities and equation
%
\begin{align}
\frac{E_y (t)}{E_{0y}} =  (cos(\tau) cos(\delta)-sin(\tau)sin(\delta))
\end{align}
%
Again using known trigonometric identities
%
\begin{align}
    \frac{E_y (t)}{E_{0y}} = \frac{E_x (t)}{E_{0x}}   cos(\delta)-\sqrt{1-\frac{E_x (t)^2}{E_{0x}^2} } sin(\delta)
\end{align}
%
Rearranging and squaring both sides results in
%
\begin{align}
    (\frac{E_x (t)}{E_{0x}}   cos(\delta)-\frac{E_y (t)}{E_{0y}} )^2=(\sqrt{1-\frac{E_x (t)^2}{E_{0x}^2} } sin(\delta))^2
\end{align}
%
The factorization of this equation can be rearranged into the standard form of an ellipse such that
%
\begin{align}
    \frac{E_x (t)^2}{E_{0x}^2} +\frac{E_y (t)^2}{E_{0y}^2} -2 \frac{E_x E_y}{E_{0x} E_{0y} } cos(\delta)=sin^2 (\delta)
\end{align}
%
And is graphed as seen in Figure 2.2.
%
\begin{figure}
    \begin{center}
        \makebox[\textwidth]{\includegraphics[scale=0.5]{/Sources/Background/Nature_of_Light/polarization-ellipse-binary-v2.png}}
    \end{center}
    \caption{Polarization Ellipse}
    \label{fig:polarization}
\end{figure}
%

Due to the restraints of modern optical sensors, it is not possible to directly measure the polarization ellipse, for a light beam, at any instant in time.  Taking the time average of the ellipse results in quantities that can measured by detectors in order to quantify the polarization state of an EM wave.  It is therefore necessary to derive parameters from the ellipse that can be measured.
Starting from the equation for the polarization ellipse, taking the time average of the E field results in
%
\begin{align}
    \frac{E_x (t)^2}{E_{0x}^2} + \frac{E_y (t)^2}{E_{0y}^2} - \frac{2 E_x (t) E_y (t)}{E_{0x} E_{0y} } cos(\delta)=sin^2 (\delta)
\end{align}
%
the time averages are calculated as
%
\begin{align}
    <E_x (t)^2> = lim_{T \rightarrow \infty} \int_{0}^{2\pi} E_{0x} cos(\tau)  d\tau=\frac{1}{2} E_{0x}^2
\end{align}
%
and similarly
%
\begin{align}
    <E_y (t)^2>  = \frac{1}{2} E_{0y}^2 \\
	<E_x (t) E_y (t)>  =  \frac{1}{2} E_{0x} E_{0y} cos(\delta)
\end{align}
%
substitution into equation and completing the square results in
%
\begin{align}
    (E_{0x}^2+E_{0x}^2 )^2-(E_{0x}^2-E_{0x}^2 )^2-(2E_{0x}^2 E_{0x}^2  cos(\delta) )^2=(2E_{0x} E_{0y}  sin(\delta))^2
\end{align}
%
The terms of this equation represent the polarization state of a wave in relation to the x and y intensities and relative phase delay between the two components.
These quantities are known as the Stokes parameters and describe the state of polarization and are often represented as a vector,
%
\begin{align}
    \vec{S} =
    \begin{bmatrix}
        S_0 \\
        S_1 \\
        S_2 \\
        S_3
    \end{bmatrix}
    =
    \begin{bmatrix}
        E_{0x}^2+E_{0y}^2 \\
        E_{0x}^2-E_{0y}^2 \\
        2E_{0x}^2 E_{0y}^2 cos(\delta) \\
        2E_{0x} E_{0y}  sin(\delta)
    \end{bmatrix}
\end{align}
%
The degree of polarization for an EM wave is the magnitude of the Stokes vector such that
%
\begin{align}
    \alpha =DOP=  \frac{\sqrt{S_1^2+S_2^2+S_3^2 }}{S_0}
\end{align}
%
and ranges from 0 for unpolarized light, to 1 for completely polarized light.  It is possible to show the polarized and unpolarized intensities as individual components
summed together as
%
\begin{align}
    S=S_P+S_U=
    \begin{bmatrix}
        S_0 \\
        S_1 \\
        S_2 \\
        S_3
    \end{bmatrix}
    +(1-DOP)
    \begin{bmatrix}
        1 \\
        0 \\
        0 \\
        0
    \end{bmatrix}
\end{align}
%
The degree of polarization for the linear (DOLP) and circular polarization (DOCP) can specifically be quantified as
%
\begin{align}
    DOLP=  \frac{\sqrt{S_1^2+S_2^2 }}{S_0} \\
    DOCP=  \frac{S_3}{S_0}
\end{align}
%
Note the unpolarized light is represented as
%
\begin{align}
    S=
    \begin{bmatrix}
        S_0 \\
        S_1 \\
        S_2 \\
        S_3
    \end{bmatrix}
    =
    \begin{bmatrix}
        1 \\
        0 \\
        0 \\
        0
    \end{bmatrix}
\end{align}
%
The Stokes parameters can be graphed on a unit sphere, known as the Poincare sphere.  The sphere plots the radial coordinates describing ellipticity and eccentricy of the polarization ellipse as angles of
%
\begin{align}
    Ellipicity = \frac{S_3}{S_0+\sqrt{S_1^2+S_2^2 }}
\end{align}
\begin{align}
    Eccentricity= \sqrt{1-Ellipticity^2}
\end{align}
%
The ellipticity of the polarization ellipse varies from 0, for linearly polarized light, to 1 for purely circular polarization \cite{chipman}.
%
For	graphical representation, Stokes vectors can be plotted on a 3-dimensional sphere known as the Poincare sphere. The sphere is only capable of showing the polarized portion of the EM wave.  Prior to normalization, if the EM wave is not fully polarized, the intensity of the polarized beam must be normalized in relation to the total beam intensity.
%
\begin{figure}[!htb]
    \begin{center}
        \makebox[\textwidth]{\includegraphics[scale=0.6]{/Sources/Background/Nature_of_Light/POINCARE-binary.png}}
    \end{center}
    \caption{Poincare Sphere}
    \label{fig:polarization}
\end{figure}
%
The zenith angle of the polarization ellipse, represented in Figure 2.3 as $2\chi$ is found to be related to the parameters of the polarization ellipse by
%
\begin{align}
    sin2\chi = \frac{2E_{0x}2E_{0y}sin\delta}{E_{0x}^2+E_{0y}^2}\qquad -\pi / 4 \leq \chi \leq \pi / 4
\end{align}
%
The azimuth angle $2\psi$ is defined in relation to the parameters of the polarization ellipse as
%
\begin{align}
    tan2\psi = \frac{2E_{0x}2E_{0y}cos\delta}{E_{0x}^2-E_{0y}^2}\qquad 0 \leq \psi \leq \pi
\end{align}
%
For a given state of the polarization ellipse, the angles for plotting the corresponding polarization state on the Poincare sphere can be found using these equations\cite{spieellipse}.

%%%%%%%%%%%%%%%%%%%%%%%%%%%%%%%%%%%%%%%%%%%%%%%%%%%%%%%%%%%
\subsection{Jones Vector Representation}
%%%%%%%%%%%%%%%%%%%%%%%%%%%%%%%%%%%%%%%%%%%%%%%%%%%%%%%%%%%
    %need subsub section on Jones
    % need subsub section on Devices
For the special case of fully polarized EM waves, DOP = 1, the polarization of the beam can be described by a 2x 1 complex vector known as a Jones vector.  The Jones vector relies on the fact that the polarization state of a beam depends only on its relative $X$ and $Y$ intensities, as well as the phase delay between each respective component.

Converting the equation for the electric component of the EM wave into a phasor makes it easy to see the parameters that determine the beam's polarization.  A phasor represents a sinusoidal wave with a constant frequency.  The Jones vectors can be formulated from this representation as
%
\begin{align}
    \underline{\hat{E}}(z)=(\vec{i_x} E_{0x} e^{j\delta_x}+\vec{i_y} E_{0y} e^{j\delta_y })e^{-kz}
\end{align}
%
since the polarization depends on the amplitude and phase difference of the $X$ and $Y$ components, the Jones vector is formally written as
%
\begin{align}
    \underline{J} =
    \begin{bmatrix}
        E_{0x} e^{j\delta_x} \\
        E_{0y} e^{j\delta_y }
    \end{bmatrix}
\end{align}
%
Since only the relative phase differences matter it is common to denote $\delta=\delta_y-\delta_x$.  The vector is also normalized by dividing by its magnitude,
%
\begin{align}
    \underline{J} =
    \frac{1}{\sqrt{E_{0x}^2 + E_{0y}^2}}
    \begin{bmatrix}
        E_{0x} e^{j\delta_x} \\
        E_{0y} e^{j\delta_y }
    \end{bmatrix}
\end{align}
%
An angle can then be defined such that
%
\begin{align}
    tan(\psi) = \frac{E_0y}{E_0x}
\end{align}
%
The Jones vector can then be written in terms of a single angle
%
\begin{align}
    \underline{J_{\delta}}(\psi) =
    \begin{bmatrix}
        cos(\psi) \\
        sin(\psi)e^{j\delta}
    \end{bmatrix}
\end{align}
%
General states of linear polarization are represented as
%
\begin{align}
    \underline{J_0}(\psi) =
    \begin{bmatrix}
        cos(\psi) \\
        sin(\psi)
    \end{bmatrix}
\end{align}
%
where $\psi$ is any angle in relation to the X axis.  Circular polarization is represented as
%
\begin{align}
    RCP: \underline{J}_{\frac{\pi}{2}} = \frac{1}{\sqrt{2}}
    \begin{bmatrix}
        1 \\
        j
    \end{bmatrix} \\
    LCP: \underline{J}_{-\frac{\pi}{2}} = \frac{1}{\sqrt{2}}
    \begin{bmatrix}
        1 \\
        -j
    \end{bmatrix}
\end{align}
%
\subsubsection{Optical Devices}
Polarization can be naturally occurring, such as in the case of skylight, or it can be created by passing light through an optical device such as a linear polarizer or a quarter wave plate.  Jones vectors are useful for describing the polarization state of an EM wave, while Jones matrices describe nondepolarizing optical devices and the transformation of pure incident polarization states through them.

A linear polarizer is a device that transmits linear polarization states for incident light beams that are aligned with their transmission axis (TA) of the polarizer [14].  For example, if horizontally polarized light is passed through a polarizer with a $TA = 90^{\circ}$, all of the incident light will be extinguished.  In practice all of the light is not completely extinguished and there are often spectral differences to the response of polarizers.

Since linear polarizers block light that is orthogonal to the TA, it can be shown that the general equation for a linear polarizer is such that
%
\begin{align}
    \underline{J}_{in}(\psi + \frac{\pi}{2}) =
    \begin{pmatrix}
        cos(\psi + \frac{pi}{2}) \\
        sin(\psi + \frac{pi}{2})
    \end{pmatrix}
    =
    \begin{pmatrix}
        -sin(\psi) \\
        cos(\psi)
    \end{pmatrix}
\end{align}
%
and
\begin{align}
    \underline{J}_{out} =
    \begin{pmatrix}
        0 \\
        0
    \end{pmatrix}
\end{align}
%
The general equation for Jones interaction with a linear polarizer is
%
\begin{align}
    P(\psi)\underline{J}_{in} = \underline{J}_{out}
\end{align}
\begin{align}
    \begin{pmatrix}
        a & b \\
        c & d
    \end{pmatrix}
    \begin{pmatrix}
        -sin(\psi) \\
        cos(\psi)
    \end{pmatrix}
    =
    \begin{pmatrix}
        0 \\
        0
    \end{pmatrix}
\end{align}
\begin{align}
    -asin(\psi) + bcos(\psi) = 0 \rightarrow atan(\psi) \\
    -csin(\psi) = dcos(\psi) = 0 \rightarrow ctan(\psi)
\end{align}
%
When the incident polarization state is aligned with the polarizers TA, it must also be true that the incident beam goes through the device unchanged.  Therefore,
%
\begin{align}
    \underline{J}_{in}(\psi) =
    \begin{pmatrix}
        cos(\psi) \\
        sin(\psi)
    \end{pmatrix} \\
    \underline{J}_{out}(\psi) =
    \begin{pmatrix}
        cos(\psi) \\
        sin(\psi)
    \end{pmatrix}
\end{align}
\begin{align}
    acos(\psi) + bsin(\psi) = cos(\psi) \\
    ccos(\psi) + dsin(\psi) = sin(\psi)
\end{align}
%
substituting in for previous values of b and d give
%
\begin{align}
    a = \frac{cos(\psi)}{cos(\psi) + tan(\psi)sin(\psi)} = cos^2\psi
\end{align}
\begin{align}
    b = atan(\psi) = sin(\psi)cos(\psi)
\end{align}
\begin{align}
    c = \frac{sin(\psi)}{cos(\psi) + tan(\psi)sin(\psi)} = sin(\psi)cos(\psi)
\end{align}
\begin{align}
    d = ctan(\psi) = sin^2\psi
\end{align}
%
The general form of a linear polarizer with transmission axis angle ? from the X axis is
%
\begin{align}
    P(\psi) =
    \begin{pmatrix}
        cos^2\psi & sin(\psi)cos(\psi) \\
        sin(\psi)cos(\psi) & sin^2\psi
    \end{pmatrix}
\end{align}
%
The intensity of light emerging from a polarizer is governed by Malus law,
%
\begin{align}
    I = I_0cos^2(\theta_i)
\end{align}
%
where $I$ is the intensity of the exiting beam, $I_0$ is the intensity of the incident beam and $\theta_i$ is the angle between the incident polarization state, and the angle of the polarizer.   For incident unpolarized light the equation becomes $I / I_0 = \frac{1}{2}$.  Therefore, the maximum transmittance for an unpolarized beam of light through a polarizer is 50%.  This number is significantly lower for polaroid film based polarizers and can cause potential problems with detector sensitivity for low lighting conditions.

Wave plates create a phase delay between the fast and slow axis of incident linearly polarized light.  Its generalized Jones matrix form can be denoted with a relative phase delay $\delta=\delta_y-\delta_x$,
\begin{align}
    C(\delta) =
    \begin{pmatrix}
        1 & 0 \\
        0 & e^{-j\delta}
    \end{pmatrix}
\end{align}
%
Two common wave plates are the half wave plate and the quarter wave plate.  These produce a delay of $\pi$ and $\pi/2$ respectively.

The noobelectric python package has a module for dealing with these types of problems and it is easy to do Jones optical calculations for purely polarized light beams.  An example for a linear polarizer combined with a quarter wave plate can be created and various input polarization states into the system as shown below.
\begin{lstlisting}
>>> from noobee.jones import j, lp
>>> j_in = j(1,0)
>>> lp_1 = lp(0)
>>> lp_1 * j_in
matrix([[ 1.],
        [ 0.]])
\end{lstlisting}

%%%%%%%%%%%%%%%%%%%%%%%%%%%%%%%%%%%%%%%%%%%%%%%%%%%%%%%%%%%
\subsection{Mueller Matrices}
%%%%%%%%%%%%%%%%%%%%%%%%%%%%%%%%%%%%%%%%%%%%%%%%%%%%%%%%%%%
Interactions with materials that can create depolarization and model partially polarized input and outputs cannot be handled by Jones calculus.  For these problems, Mueller Matrices are used to model the polarization of light with varying degrees of polarization as it interacts with a material.  These interactions are modeled with equation [CHECK add system diagram here with boxes] and shown in Figure.
\begin{align}
    \mathbf{S}_{out} = \mathbf{M}\mathbf{S}_{in}
\end{align}
%
When EM waves interact with optically active materials, their state of polarization may change.  The characteristics of a material which changes the amplitude or phase of the $x$ or $y$ component for an incident EM wave, i.e. the  polarization, is defined by its Mueller matrix.  Mueller matrices determine how input Stokes' vectors change upon interaction with a material. They are defined as
%
\begin{align}
    \mathbf{M} =
    \begin{bmatrix}
        m_{00} & m_{01} & m_{02} & m_{03} \\
        m_{10} & m_{11} & m_{12} & m_{13} \\
        m_{20} & m_{21} & m_{22} & m_{23} \\
        m_{30} & m_{31} & m_{32} & m_{33}
    \end{bmatrix}
\end{align}
%
and describe the diattenuation, depolarization, and retardance of a materials' polarization response to an input EM beam.
%The general equation for polarization material interaction is
\begin{figure}
    \begin{center}
        \makebox[\textwidth]{\includegraphics[scale=0.6]{/Sources/Background/Nature_of_Light/system_stokes.png}}
    \end{center}
    \caption{Stokes-Mueller System Diagram}
    \label{fig:polarization}
\end{figure}
%\begin{align}
%    \begin{bmatrix}
%        S_0 \\
%        S_1 \\
%        S_2 \\
%        S_3
%    \end{bmatrix}
%        m_{00} & m_{01} & m_{02} & m_{03} \\
%        m_{10} & m_{11} & m_{12} & m_{13} \\
%        m_{20} & m_{21} & m_{22} & m_{23} \\
%        m_{30} & m_{31} & m_{32} & m_{33}
%    \end{bmatrix}
%    \begin{bmatrix}
%        S_0^\prime \\
%        S_1^\prime \\
%        S_2^\prime \\
%        S_3\prime
%    \end{bmatrix}
%\end{align}

%[CHECK these get bolded]
\underline{Diattenuation} – the two attenuations of orthogonal polarization states \cite{chipman}

\underline{Retardance} – the phase difference between two orthogonal polarization states \cite{giakos}

\underline{Depolarization} – a process where polarized light becomes unpolarized \cite{giakos}

It has been shown that these parameters can be found by determining a sample's corresponding Mueller Matrix. The scalar parameters are mathematically defined as,
\begin{align}
    \mathbf{Diattenuation} = \frac{T_{max} - T_{min}}{T_{max} + T_{min}}
\end{align}
\begin{align}
    \mathbf{Retardance} = \delta = \frac{2\pi(n_1 - n_2)t}{\lambda}
\end{align}
\begin{align}
    \mathbf{Depolarization} = 1 - DOP
\end{align}
were $T_{max}$ and $T_{min}$ are the intensity transmittances through a polarizer, $n_1$, $n_2$ and $t$ are the refractive indices and thickness of a retarder \cite{chipman}.
[CHECK explain these equations or remove them]

For nondepolarizing materials, their MM can be converted into Jones matrices.   Examples of Mueller Matrices for common optical elements can be found in \cite{chipman}.  The general form of a linear polarizer with transmission axis at 0 degrees to the $x$ axis is
%
\begin{align}
    \mathbf{M} =
    \begin{bmatrix}
        q + r & q - r & 0 & 0 \\
        q-r & q+r & 0  & 0 \\
        0 & 0 & 2\sqrt{qr} & 0 \\
        0 & 0 & 0 & 2\sqrt{qr}
    \end{bmatrix}
\end{align}
%
where $q$ and $r$ are the attenuation coefficients for the $x$ and $y$ axis [TODO relate this to the later equation showing transmissiona and reflection].

%%%%%%%%%%%%%%%%%%%%%%%%%%%%%%%%%%%%%%%%%%%%%%%%%%%%%%
\subsubsection{Mueller Matrix Decomposition}
%%%%%%%%%%%%%%%%%%%%%%%%%%%%%%%%%%%%%%%%%%%%%%%%%%%%%%

The effects of diattenuation, depolarization and retardance can be found to be represented as subsets of the Muller matrix through its decomposition such that the result is
\begin{align}
    \mathbf{M} = m_{00}
    \begin{bmatrix}
        1 & \mathbf{D}^T \\
        \mathbf{P} & \mathbf{m}
    \end{bmatrix}
\end{align}
The derivation of the MM decomposition was made known by Lu and Chipman and is reproduced in \cite{polarizedlight}. $\mathbf{D}^T$ is the diattenutation vector that describes the amount of decrease in overall polarization for each set of orthogonal polarization states. The m matrix represents the retardance of a material.

$\mathbf{P}$ is the polarizance vector, and describes the amount of light that becomes polarized when unpolarized light is incident.  It is analogous to the effect of depolarization.  This vector can be measured by detecting the output Stokes vector of a material when unpolarized light is incident.  This effect is only evident in materials that create polarization.

%%%%%%%%%%%%%%%%%%%%%%%%%%%%%%%%%%%%%%%%%%%%%%%%%%%%%%%%%%%
\subsection{Reflection and Transmission}
%%%%%%%%%%%%%%%%%%%%%%%%%%%%%%%%%%%%%%%%%%%%%%%%%%%%%%%%%%%

When light interacts with two materials that have different indexes of refraction, the incident beam is reflected and transmitted according to Fresnel’s equations.

The law of reflection states
%
\begin{align}
    \theta_i = \theta_r
\end{align}
%
The transmission is derived directly from Snells equation,
%
\begin{align}
    n_1sin(\theta_i) = n_2sin(\theta_2)
\end{align}
%
Note that part of the transmitted spectra may be absorbed, although this is not considered in these equations.

Two main scenarios are often presented when demonstrating the principles which guide the reflected and transmitted rays; when the incident electromagnetic wave is polarized perpendicular to the plane of incidence, and when the wave is polarized parallel to it.  The perpendicular polarized wave is often denoted S or TE for transverse electric, while the parallel scenario is often denoted P or TM for transverse magnetic.

%Figure 4-Fresnel reflection and transmission. reproduced here without permission%

The intensity of the reflected waves for the S and P polarized case are
%
\begin{align}
    \mathbf{R_S} = |\frac{Z_2cos(\theta_i)-Z_1cos(\theta_t)}{Z_2cos(\theta_i)+Z_1cos(\theta_t)}|^2
\end{align}
\begin{align}
    \mathbf{R_P} = |\frac{Z_2cos(\theta_t)-Z_1cos(\theta_i)}{Z_2cos(\theta_t)+Z_1cos(\theta_i)}|^2
\end{align}
%
where Z is the wave impedance for medium 1 and 2.  The power coefficients for transmission are then derived by following the law of conservation of energy such that,
%
\begin{align}
    T_S = 1 - R_S \\
    T_P = 1 - R_P
\end{align}
%
The Brewster angle is a special case where the $P$ polarization state is completely transmitted and no reflection of the TM wave occurs.  The reflected ray is therefore completely $S$ polarized since $R_P$ is zero and $R_S$ is a nonzero intensity.  For perfect air glass interactions typically considered, this angle is approximately 55 degrees.  Figure 5 shows the reflection and transmission coefficients versus incident angle for these two media.

%Figure 5-Transmission and reflection coefficients for air and glass. Reproduced here without permission.%

The Mueller matrix formulation for reflection and transmission reduces to the form of a linear polarizer for ideal surfaces.  It has been shown in [14] that the equation for a reflected beam off of a perfectly smooth dielectric surface is
%
\begin{align}
    \begin{bmatrix}
        S_{0r} \\
        S_{1r} \\
        S_{2r} \\
        S_{3r}
    \end{bmatrix}
    =
% TODO
%    \frac{1}{2}(\frac{tan(\theta_{_})}{sin(\theta_{+})})^2
%    \begin{bmatrix}
%        cos^2(\theta_{_}) + cos^2(\theta_+) & cos^2(\theta_{_}) - cos^2(\theta_+) & 0 & 0 \\
%        cos^2(\theta_{_}) - cos^2(\theta_+) & cos^2(\theta_{_}) - cos^2(\theta_+) & 0 & 0 \\
%        0 & 0 & -2cos(\theta_{_})cos(\theta_+) & 0 \\
%        0 & 0 & 0 & -2cos(\theta_{_})cos(\theta_{+})
%    \end{bmatrix}
\end{align}
%
where $\theta_{\pm}=\theta_i \pm \theta_r$. This is identical to the form of a linear diattenuator or polarizer.

For incident unpolarized light the equation simplifies to
%

%
Therefore, for ideal reflective surfaces, at the Brewster angle, light will be completely polarized perpendicular to the plane of incidence.

Imperfect, non-ideal surfaces have the ability to reflect, transmit and absorb incident electromagnetic radiation.  The outcome of these interactions are related to the physiological makeup of the material, as well as its surface topology.  Multiple scattering mechanisms can be at work within a system, and numerous models have been attempted to balance the tradeoffs between practical realizability for measurements and accurate representation of scattering mechanisms [22, 23].  Only some of these models attempt to deal with the polarization of the incident, reflected, and transmitted beams.

%%%%%%%%%%%%%%%%%%%%%%%%%%%%%%%%%%%%%%%%%%%%%%%%%%%%%%%%%%%
\subsection{Scattering Mechanisms}
%%%%%%%%%%%%%%%%%%%%%%%%%%%%%%%%%%%%%%%%%%%%%%%%%%%%%%%%%%%

Fresnel’s equations provide an explanation for light reflected and transmitted for ideal surfaces.  This is not the case with most man made and natural materials.  Therefore, more complex mechanisms must be considered when dealing with real world radiation scattering problems.  It has become popular in the field of remote sensing to denote the additional types of interactions as volume scattering and multiple scatter interaction.  Single scattering mechanisms are those governed solely by Fresnel’s equations.  The combination of these scattering mechanisms create the diffuse and specular components of reflection.

Single scattering mechanisms create a portion of reflectance known as specular reflectance.  They are often denoted as Type A photons in remote sensing models.  These interactions are highly polarized perpendicular to the plane of incidence as previously discussed. For perfectly smooth dielectrics, this is the dominate scattering mechanism.

Volume scattering occurs when light is absorbed by a material and is readmitted in all directions, including back towards the surface of the material.  They are denoted Type B photons.  Transmittance of this energy back into the first medium obeys the laws of Fresnel’s equations, although the indices of refraction are reversed.  This mechanism accounts for absorption and other higher level light matter interactions, not explained solely by Fresnel’s equations.

Multiple scattering occurs when either Type A or Type B photons interact with the material surface more than once when either being reflected or re transmitted out of the material.  These are denoted Type C photons \cite{schott}.  Figure 6 shows each type of interaction.
%
\begin{figure}[!htb]
    \begin{center}
        \makebox[\textwidth]{\includegraphics[scale=0.75]{/Sources/Background/Nature_of_Light/photon-interactions-wo.png}}
    \end{center}
    \caption{Various Types of Photon Interactions}
    \label{fig:scattering}
\end{figure}
%
In general type A photons create specular highlights from surfaces that are smooth.  Type B and C photons become more prevalent as a surface becomes rougher.  In most real world applications, surfaces are neither purely specular or purely diffuse.

Surfaces that are perfectly smooth dielectrics are often considered to be purely specular reflectors of light.  Incident energy is transmitted in an idealized single ray of light from the surface.  Specular reflectors are single scattering mechanisms and result in purely polarized light due to the governance of Fresnel’s equations and are denoted as type A photons.

In simple models, rough surfaces can be viewed as purely diffuse reflectors that scatter incident light equally in all directions.  Perfect diffuse surfaces are known as Lambertian surfaces.  It has been assumed that the diffuse portion of light is unpolarized due to random nature of internal reflections [\cite{specularclass}, \cite{grant}]. Bidirectional Reflectance Distribution Functions (BRDF) have been created to model the variety of surface interactions in order to handle the non ideal case of rough surfaces. (TODO in its simplest form rotational symmetry is assumed around the point of incidence and the equation becomes)
\begin{figure}[!htb]
    \begin{center}
        \makebox[\textwidth]{\includegraphics[scale=0.45]{/Sources/Background/Nature_of_Light/diffuse-specular.png}}
    \end{center}
    \caption{Diffuse and Specular Scattering}
    \label{fig:scattering}
\end{figure}

The notion of a surface being rough, smooth, fine or coarse come with the connotation of touch and the feeling of a materials' surface.  They are textures.

